% Chapter 1

\chapter{Introduction} % Main chapter title

\label{Introduction} % For referencing the chapter elsewhere, use \ref{Chapter1} 

\lhead{Chapter 1. \emph{Introduction}} % This is for the header on each page - perhaps a shortened title

%----------------------------------------------------------------------------------------
\section{Classical Kicked Rotor}

In the classical picture, we can write the Hamiltonian of single kicked top as 

\begin{equation}
{H}=\frac{{J}^2}{2{I}} +\sum_{n} {V}\delta(t-nT)
\end{equation}

where ${J}$ is the angular momentum of the top, ${I}$ is the moment of inertia, and ${V}$ is the kicked potential.

If we try to construct a system having two kicked tops coupled to each other, the corresponding Hamiltonian of the system can be written as

\begin{equation}
H=\frac{{J_1}^2}{2{I_1}}+\frac{{J_2}^2}{2{I_2}}+\sum_{n} (V_1+V_2+V_{12})\delta(t-nT)
\end{equation}

where $V_{12}$ is an interaction potential that couples the dynamics of the system.

\section{Quantum Kicked Rotor}
In the quantum picture, we replace the angular momentum $J$'s by the corresponding angular momentum operators $\hat{J}$, and the Hamiltonian for coupled tops can be written as follows

\begin{equation}
H=\frac{p_1}{2j}{J_1}^2+\frac{p_2}{2j}{J_2}^2+\frac{\epsilon}{j}J_{z_1}J_{z_2}
\end{equation} 

where $\epsilon$ is the coupling strength, and $j$ are the eigenvalues of the $J^2$ operator. The value of $\epsilon$ determines the entaglement properties of this coupled system.

\section{Measures of Entaglement}
Entaglement can be defined as the non classical correlation between two spatially separated subsystems \cite{jnbbig}. Entaglement can be measured calculating the von Neumann entropy of the Reduced Density Matrix of the Hamiltonian. The von Neumann entropy $S_v$ is given by

\begin{equation}
S_v=-\sum_{n}\lambda_i\log\lambda_i
\end{equation}

where $\lambda_i$'s are the eigenvalues of the reduced density matrix.